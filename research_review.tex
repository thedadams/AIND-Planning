\documentclass{article}

\usepackage{fancyhdr}
\usepackage{titlesec}
\usepackage{amsmath}

\linespread{1.5}
\setlength{\parskip}{1em}
\setlength{\textwidth}{6.5in}
\setlength{\textheight}{9in}
\setlength{\headheight}{.5in}
\setlength{\topmargin}{0in} \addtolength{\topmargin}{-\headheight}
\addtolength{\topmargin}{-\headsep}
\setlength{\oddsidemargin}{0in}
\setlength{\evensidemargin}{0in}
\setlength{\parindent}{0em}

\newcommand{\GRAPHPLAN}{\texttt{GRAPHPLAN}}
\newcommand{\STRIPS}{\texttt{STRIPS}}

\pagestyle{fancy}\lhead{\textit{Game Tree Searching by Min/Max Approximation} by Ronald L. Rivest} \rhead{}
\chead{} \lfoot{} \rfoot{} \cfoot{}

\begin{document}
Artificial intelligence planning has had many applications over the years. The techniques in this realm have been used in classical planning, control theory, and theorem proving.\cite{AMIA} There have obviously been numerous advancements, both big and small, that have gotten us to this point. In this research review, we focus on three such advancements: the \STRIPS{} algorithm, the \GRAPHPLAN{} algorithm, and optimizations to \GRAPHPLAN{}.

The first big advancement in this area was the \STRIPS{} algorithm. \STRIPS{} was developed as planning algorithm for a mobile robot that could navigate and push objects around in a multi-room environment\cite{StripsRevisit}. Although this was just one part of the ``Skakey the robot'' project, it established the foundation of classical planning terminology. For example, all the propositional logic we used in the project was originally used in the \STRIPS{} algorithm\cite{Weld1999}. However, this algorithm had limitations: it assumed that actions could be applied one at any time, that nothing changed except as a result of the actions, and that actions were instantaneous\cite{StripsRevisit}. Of course these assumptions are not a realistic model of the world we occupy and it was shown that \STRIPS{} could not solve some relatively simple problems\cite{AMIA}. This led to some other advancement that eventually lead us to \GRAPHPLAN{}.



\bibliography{biblio}{}
\bibliographystyle{plain}
\end{document}
